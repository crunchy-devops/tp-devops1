%%%%%%%%%%%%%%%%%%%%%%%%%%%%%%%%%%%%%%%%%
% Lachaise Assignment
% LaTeX Template
% Version 1.0 (26/6/2018)
%
% This template originates from:
% http://www.LaTeXTemplates.com
%
% Authors:
% Marion Lachaise & François Févotte
% Vel (vel@LaTeXTemplates.com)
%
% License:
% CC BY-NC-SA 3.0 (http://creativecommons.org/licenses/by-nc-sa/3.0/)
% 
%%%%%%%%%%%%%%%%%%%%%%%%%%%%%%%%%%%%%%%%%

%----------------------------------------------------------------------------------------
%	PACKAGES AND OTHER DOCUMENT CONFIGURATIONS
%----------------------------------------------------------------------------------------

\documentclass{article}
\usepackage{graphicx}
\input{structure.tex} % Include the file specifying the document structure and custom commands

%----------------------------------------------------------------------------------------
%	ASSIGNMENT INFORMATION
%----------------------------------------------------------------------------------------

\title{Plateforme de CI/CD} % Title of the assignment

\author{Hervé Meftah\\ \texttt{hmeftah@system-dev.io}} % Author name and email address

\date{\today} % University, school and/or department name(s) and a date

%----------------------------------------------------------------------------------------

\begin{document}

\maketitle % Print the title

%----------------------------------------------------------------------------------------
%	INTRODUCTION
%----------------------------------------------------------------------------------------

\section*{Introduction} % Unnumbered section

Ce document répond à la spécification suivante:
%\begin{info} % Information block
%	Définir comment déployer, dans Kubernetes, une instance d’un projet à chaque fois qu’une pull-request est envoyée et
%	mettre en place l’automatisation qui le fait.
%\end{info}

\begin{info}[Expression des besoins]
	Définir comment déployer, dans Kubernetes, une instance d’un projet à chaque fois qu’une pull-request est envoyée et
	mettre en place l’automatisation qui le fait.
\end{info}



% Math equation/formula
%\begin{equation}
%	I = \int_{a}^{b} f(x) \; \text{d}x.
%\end{equation}
%The following image depicts the Kubernetes objects involved
%\begin{figure}[h!]
%	\caption{kubernetes objects}
%	\centering
%	\includegraphics[width=\textwidth]{../screenshot/glusterfs-postgresql.jpg}
%\end{figure}

%----------------------------------------------------------------------------------------
%	PROBLEM 1
%----------------------------------------------------------------------------------------

\section{Présentation de la plateforme CI/CD } % Numbered section

%------------------------------------------------

\subsection{Le schéma}

Le schéma ci-dessus décrit les différents composants qui seront utilisés pour déployer les applications dans Kubernetes.
\begin{figure}[h!]
\caption{plateforme CI/CD}
	\centering
	\includegraphics[scale=0.35]{../screenshot/schema.png}
\end{figure}
Tous ces composants sont sous Docker, communicant les uns aux autres par le DNS virtuel de Docker. Des disques/volumes sont
utilisés pour permettre la persistance des données entre les redémarrages de la plateforme. \\
Github est le gestionnaire de source, il gere les pull-request du \textit{master-repo} qui sont initiés par le repo d'un
utilisateur connecté a \textit{user-repo}. Des \textit{webhooks} sont placés comme déclencheurs de la chaine d'intégration
continue (CI/CD) suite à la validation, par \textit{le code reviewer}, des corrections ou ajouts faits dans le code source. \\
L'image du container Jenkins est construite à partir d'un Dockerfile s'appuyant sur la version LTS de Jenkins. Par défaut,
elle inclus:  le langage Go, la derniere version de python (3.9) et le gestionnaire de build Maven. A l'initialisation le
mot de passe administrateur est présent dans les logs du container. \\
Pour des buts de formation, les plugins Jenkins ne sont installés par defaut. Une installation sans plugin est préférable
pour bien s'approprier les concepts de Jenkins. \\
Le container SonarQube scanne le code source pour effectuer un controle qualité statique, il vérifie les duplications,
le code ``mort'', la couverture des tests unitaires.
En sortie de SonarQube, une tache Jenkins effectue un \textit{docker build} pour injecter le code source dans une image
docker. \\
Le container Nexus, correspond a un Artifactory, archive les images, les sources intermediaires pour un usage ultérieur.
Jmeter et Selenium sont des outils pour respectivement tester les montees en charge et les tests fonctionnels a travers
l'interface Web ou GUI de l'application. \\
L'automatisation de cette chaine d'intégration, pilotée par Jenkins, inclus Ansible et la version open-source Tower: AWX.
Le provisionning d'infrastructure est fait avec Ansible, et en fin de chaine, Ansible se connecte a la plateforme Kubernetes
pour déployer les artifacts validés par les tests et stockés dans Nexus.

%------------------------------------------------

\section{Installation de la platforme}
\subsection{Pre-requis}
Quelques outils doivent etre installes en premier pour faciliter la gestion de la plateforme.
\begin{commandline}
	\begin{verbatim}
		## Pre-requis pour Centos 7
		sudo yum -y update   # update all packages
		sudo yum -y install git wget   # install git and wget
		sudo yum -y install epel-release  # added extra packages
		sudo yum -y install htop iotop iftop  # added monitoring tools
		//Fork
		//     https://github.com/crunchy-devops/tp-devops1.git
		git clone your personnal repository of tp-devops1
		git clone https://github.com/<your-repo>/tp-devops1.git
		cd tp-devops1
	\end{verbatim}
\end{commandline}

\subsection{Installation de Docker}

In malesuada ullamcorper urna, sed dapibus diam sollicitudin non. Donec elit odio, accumsan ac nisl a, tempor imperdiet eros. Donec porta tortor eu risus consequat, a pharetra tortor tristique. Morbi sit amet laoreet erat. Morbi et luctus diam, quis porta ipsum. Quisque libero dolor, suscipit id facilisis eget, sodales volutpat dolor. Nullam vulputate interdum aliquam. Mauris id convallis erat, ut vehicula neque. Sed auctor nibh et elit fringilla, nec ultricies dui sollicitudin. Vestibulum vestibulum luctus metus venenatis facilisis. Suspendisse iaculis augue at vehicula ornare. Sed vel eros ut velit fermentum porttitor sed sed massa. Fusce venenatis, metus a rutrum sagittis, enim ex maximus velit, id semper nisi velit eu purus.

%\begin{center}
%	\begin{minipage}{0.5\linewidth} % Adjust the minipage width to accomodate for the length of algorithm lines
%		\begin{algorithm}[H]
%			\KwIn{$(a, b)$, two floating-point numbers}  % Algorithm inputs
%			\KwResult{$(c, d)$, such that $a+b = c + d$} % Algorithm outputs/results
%			\medskip
%			\If{$\vert b\vert > \vert a\vert$}{
%				exchange $a$ and $b$ \;
%			}
%			$c \leftarrow a + b$ \;
%			$z \leftarrow c - a$ \;
%			$d \leftarrow b - z$ \;
%			{\bf return} $(c,d)$ \;
%			\caption{\texttt{FastTwoSum}} % Algorithm name
%			\label{alg:fastTwoSum}   % optional label to refer to
%		\end{algorithm}
%	\end{minipage}
%\end{center}

Fusce varius orci ac magna dapibus porttitor. In tempor leo a neque bibendum sollicitudin. Nulla pretium fermentum nisi, eget sodales magna facilisis eu. Praesent aliquet nulla ut bibendum lacinia. Donec vel mauris vulputate, commodo ligula ut, egestas orci. Suspendisse commodo odio sed hendrerit lobortis. Donec finibus eros erat, vel ornare enim mattis et.

% Numbered question, with an optional title
%\begin{question}[\itshape (with optional title)]
%	In congue risus leo, in gravida enim viverra id. Donec eros mauris, bibendum vel dui at, tempor commodo augue. In vel lobortis lacus. Nam ornare ullamcorper mauris vel molestie. Maecenas vehicula ornare turpis, vitae fringilla orci consectetur vel. Nam pulvinar justo nec neque egestas tristique. Donec ac dolor at libero congue varius sed vitae lectus. Donec et tristique nulla, sit amet scelerisque orci. Maecenas a vestibulum lectus, vitae gravida nulla. Proin eget volutpat orci. Morbi eu aliquet turpis. Vivamus molestie urna quis tempor tristique. Proin hendrerit sem nec tempor sollicitudin.
%\end{question}

%Mauris interdum porttitor fringilla. Proin tincidunt sodales leo at ornare. Donec tempus magna non mauris gravida luctus. Cras vitae arcu vitae mauris eleifend scelerisque. Nam sem sapien, vulputate nec felis eu, blandit convallis risus. Pellentesque sollicitudin venenatis tincidunt. In et ipsum libero. Nullam tempor ligula a massa convallis pellentesque.

%----------------------------------------------------------------------------------------
%	PROBLEM 2
%----------------------------------------------------------------------------------------

\section{Implementation}

Proin lobortis efficitur dictum. Pellentesque vitae pharetra eros, quis dignissim magna. Sed tellus leo, semper non vestibulum vel, tincidunt eu mi. Aenean pretium ut velit sed facilisis. Ut placerat urna facilisis dolor suscipit vehicula. Ut ut auctor nunc. Nulla non massa eros. Proin rhoncus arcu odio, eu lobortis metus sollicitudin eu. Duis maximus ex dui, id bibendum diam dignissim id. Aliquam quis lorem lorem. Phasellus sagittis aliquet dolor, vulputate cursus dolor convallis vel. Suspendisse eu tellus feugiat, bibendum lectus quis, fermentum nunc. Nunc euismod condimentum magna nec bibendum. Curabitur elementum nibh eu sem cursus, eu aliquam leo rutrum. Sed bibendum augue sit amet pharetra ullamcorper. Aenean congue sit amet tortor vitae feugiat.

In congue risus leo, in gravida enim viverra id. Donec eros mauris, bibendum vel dui at, tempor commodo augue. In vel lobortis lacus. Nam ornare ullamcorper mauris vel molestie. Maecenas vehicula ornare turpis, vitae fringilla orci consectetur vel. Nam pulvinar justo nec neque egestas tristique. Donec ac dolor at libero congue varius sed vitae lectus. Donec et tristique nulla, sit amet scelerisque orci. Maecenas a vestibulum lectus, vitae gravida nulla. Proin eget volutpat orci. Morbi eu aliquet turpis. Vivamus molestie urna quis tempor tristique. Proin hendrerit sem nec tempor sollicitudin.

% File contents
\begin{file}[hello.py]
\begin{lstlisting}[language=Python]
#! /usr/bin/python

import sys
sys.stdout.write("Hello World!\n")
\end{lstlisting}
\end{file}

Fusce eleifend porttitor arcu, id accumsan elit pharetra eget. Mauris luctus velit sit amet est sodales rhoncus. Donec cursus suscipit justo, sed tristique ipsum fermentum nec. Ut tortor ex, ullamcorper varius congue in, efficitur a tellus. Vivamus ut rutrum nisi. Phasellus sit amet enim efficitur, aliquam nulla id, lacinia mauris. Quisque viverra libero ac magna maximus efficitur. Interdum et malesuada fames ac ante ipsum primis in faucibus. Vestibulum mollis eros in tellus fermentum, vitae tristique justo finibus. Sed quis vehicula nibh. Etiam nulla justo, pellentesque id sapien at, semper aliquam arcu. Integer at commodo arcu. Quisque dapibus ut lacus eget vulputate.

% Command-line "screenshot"
\begin{commandline}
	\begin{verbatim}
		$ chmod +x hello.py
		$ ./hello.py

		Hello World!
	\end{verbatim}
\end{commandline}

Vestibulum sodales orci a nisi interdum tristique. In dictum vehicula dui, eget bibendum purus elementum eu. Pellentesque lobortis mattis mauris, non feugiat dolor vulputate a. Cras porttitor dapibus lacus at pulvinar. Praesent eu nunc et libero porttitor malesuada tempus quis massa. Aenean cursus ipsum a velit ultricies sagittis. Sed non leo ullamcorper, suscipit massa ut, pulvinar erat. Aliquam erat volutpat. Nulla non lacus vitae mi placerat tincidunt et ac diam. Aliquam tincidunt augue sem, ut vestibulum est volutpat eget. Suspendisse potenti. Integer condimentum, risus nec maximus elementum, lacus purus porta arcu, at ultrices diam nisl eget urna. Curabitur sollicitudin diam quis sollicitudin varius. Ut porta erat ornare laoreet euismod. In tincidunt purus dui, nec egestas dui convallis non. In vestibulum ipsum in dictum scelerisque.

% Warning text, with a custom title
\begin{warn}[Notice:]
  In congue risus leo, in gravida enim viverra id. Donec eros mauris, bibendum vel dui at, tempor commodo augue. In vel lobortis lacus. Nam ornare ullamcorper mauris vel molestie. Maecenas vehicula ornare turpis, vitae fringilla orci consectetur vel. Nam pulvinar justo nec neque egestas tristique. Donec ac dolor at libero congue varius sed vitae lectus. Donec et tristique nulla, sit amet scelerisque orci. Maecenas a vestibulum lectus, vitae gravida nulla. Proin eget volutpat orci. Morbi eu aliquet turpis. Vivamus molestie urna quis tempor tristique. Proin hendrerit sem nec tempor sollicitudin.
\end{warn}

%----------------------------------------------------------------------------------------

\end{document}
